\documentclass[12pt,a4paper]{article}
\usepackage[utf8]{inputenc}
\usepackage[german]{babel}
\usepackage{amsmath}
\usepackage{amsfonts}
\usepackage{amssymb}
%\usepackage{biblatex}
\usepackage{a4wide}
%\usepackage[babel,german=guillemets]{csquotes}
%\bibliography{literatur} 
\author{Maximilian Reuter}
\title{Zusammenfassung Mathe 4 für ET SS13}
\begin{document}
\maketitle
\tableofcontents
\newpage

\part{Grundbegriffe Statistik und Wahrscheinlichkeitsrechnung}
\section{Messreihen}
\subsection{Empirische Verteilungsfunktion}
Ordnet man eine Messreihe $x_1, x_2, x_3,...,x_n$ der Größe nach, so erhält man $x_{(1)} \leq x_{(2)} \leq ... \leq x_{(n)}$. Die empirische Verteilungsfunktion lautet:
\[F(z,x_1,x_2,...,x_n) = \frac{\text{Zahl der $x_i$ mit $x_i \leq z$}}{n} = \frac{\text{max } \{\text{ i : } x_i \leq z \}}{n}\]
Vorgehensweise für den Plot:
\begin{enumerate}
\item
Alle $F(x_i)$ bilden.
\item
$F(x_i)$ als Punkte auftragen.
\item
Linien gehen von den Punkten aus waagrecht nach rechts bis auf Höhe des nächst höheren Punktes.
\end{enumerate}

\subsection{Klassenhäufigkeiten}
Für die vorgegebene Unterteilung $a_1 \leq a_2 \leq a_3 \leq a_4$ bildet man $F(a_1) , F(a_2) - F(a_1),F(a_3)-F(a_2),F(a_4)-F(a_3)$ um die empirische Verteilung in den Intervallen zwischen den Zahlen $a_i$ zu erhalten. 

\section{Lage und Streumaßzahlen}
Gegeben sei eine Messreihe $x_1, x_2,...x_n$.
\subsection{Lagemaßzahlen}
Beispiele:
\begin{itemize}
\item
Arithmetisches Mittel:
\[ \bar{x} = \frac{1}{n} (x_1 + x_2 + ... + x_n)\]
\item
Median (das Element in der Mitte):
\[\tilde{x} = \left \{ \begin{array}{cl} x_{(frac{n}{2})}, & \text{falls $n$ gerade}  \\ x_{(\frac{n+1}{2})}, & \text{falls $n$ ungerade} \end{array} \right. \]
\item
p-Quantil ($0<p<1$):
\[x_p = \left \{ \begin{array}{cl} x_{(np)}, & \text{falls $np$ ganzzahlig} \\ x_{[np]+1}, & \text{falls $np$ nicht ganzzahlig}\end{array} \right.\]
\item
$\alpha$-gestutztes Mittel ($0 < \alpha < 0,5$):
\[\bar{x}_{\alpha} = \frac{1}{n-2k} (x_{(k+1)} + ... + x_{(n-k)}), \ k = n\alpha\]
\end{itemize}

\subsection{Streuungsmaße}
Beispiele:
\begin{itemize}
\item
Empirische Varianz oder empirische Stichprobenvarianz:
\[s^2 = \frac{1}{n-1} \sum\limits_{i=1}^{n}{(x_i -\bar{x})^2} = \frac{1}{n-1} \sum\limits^{n}_{i=1}{(x_i^2-n \bar{x}^2)}\]
\item
Empirische Streuung:
\[s = \sqrt{\frac{1}{n-1} \sum\limits^{n}_{i=1}{(x_i - \bar{x})^2}}\]
\item
Spannweite:
\[ v = x_{(n)} - x_{(1)}\]
\item
Quartilabstand:
\[q = x_{0,75} - x_{0,25}\]
\end{itemize}

\subsection{Zweidimensionale Messreihen}
Siehe oben, beide Variablen $x_i$ und $y_i$ werden getrennt betrachtet. Weiterhin:
\begin{itemize}
\item
Empirische Kovarianz:
\[s_{xy} = \frac{1}{n-1} \sum\limits^{n}_{i=1}{(x_i - \bar{x})(y_i - \bar{y})} = \frac{1}{n-1} (\sum\limits^{n}_{i = 1}{x_i y_i - n \bar{x} \bar{y})}\]
\item
Empirischer Korrelationskoeffizient:
\[r_{xy} = \frac{s_{xy}}{s_x \cdot s_y}, \text{ es gilt immer $-1 \leq r_{xy} \leq 1$.}\]
\end{itemize}

\subsection{Regressionsgerade}
Die Regressionsgerade lautet:
\[y = \hat{a} \cdot x + \hat{b}\]
mit
\[\hat{a} = \frac{\sum^n_{i=1}{ x_i y_i - n \bar{x} \bar{y}}}{\sum^n_{i=1}{x^2_i - n \bar{x}^2}} = \frac{s_{xy}}{s^2_x}, \ \hat{b}=\bar{y}-a \bar{x}\]
. Die Abweichungen der Punkte $(x_i, y_i)$ in vertikaler Richtung heißen Residuen:
\[r_i = y_i -\hat{a}x_i-\hat{b}, \ i = 1,..,n\]
Residuenquadrat:
\[\sum\limits^n_{i=1}{r^2_i} = \sum\limits^n_{i=1}{(y_i - \bar{y})^2 (1-r^2_{xy})}\]
\glqq Das Vorzeichen von $r_{xy}$ gibt den Trend der Abhängigkeit der $y$-Werte von den $x$-Werten an":
\begin{itemize}
\item
$r_{xy} > 0$: Die Gerade steigt streng monoton.
\item
$r_{xy} < 0$: Die Gerade fällt streng monoton.
\item
$r_{xy} = 0$: Die Gerade verläuft horizontal.
\item
$r_{xy} = 1$ oder $-1$: Alle Punkte liegen auf der Geraden.
\end{itemize}

\section{Zufallsexperimente und Wahrscheinlichkeit}
\subsection{Zufallsexperimente}
$\Omega$ ist die Ergebnismenge (Stichprobenmenge) und $\omega$ ein einzelnes Ergebnis (Stichprobe). $A \subset \Omega$ heißen Ereignisse. \glqq Ein Ereignis $A \subset \Omega$ tritt ein, wenn ein Ergebnis $\omega \in A$ beobachtet wird."\\
$A \cup B$ tritt ein, wenn ein Ereignis $\omega$ mit $\omega \in A$ oder $\omega \in B$ ($\omega \in A \cup B$).\\
$A \cap B$ falls $\omega \in A$ und $\omega \in B$ ($\omega \in A \cap B$).

\subsection{Wahrscheinlichkeit}
Zusammenhänge:
\[P(\emptyset) = 0\]
\[0 \leq P(A) \leq 1\]
\[P(A^C) = 1 - P(A)\]
\[A \subset B \rightarrow P(A) - P(B)\]
\[P(A \cup B) = P(A) + P(B) - P(A \cap B)\]
\[P(A_1 \cup ... \cup A_2) = \sum\limits_{i=1}^n{P(A_i)}, \text{ falls $A_1,...,A_n$ paarweise unvereinbar (Additivität).}\]
Die Annahme gleicher Wahrscheinlichkeit für die Elementarereignisse $A_i$ heißt Laplace-Annahme.

\subsection{Elementare Formeln der Kombinatorik}
\begin{itemize}
\item
Geordnete Probe mit Wiederholungen: $k$-Tupel ($x_1,...,x_k$), Formel für die Anzahl der Möglichkeiten beim ziehen von $n$ Proben: $n^k$
\item
Geordnete Probe ohne Wiederholungen: $k$-Tupel ($x_1,...,x_k$), Anzahl geordneter Proben ohne Wiederholungen: $n(n-1)(n-2)...(n-k+1)$, falls $n=k$: $n!$
\item
Lotto: Man zieht $k$ zahlen aus $n$ möglichen, Reihenfolge egal ($\begin{pmatrix} n \\ k \end{pmatrix}$): $\frac{n!}{k!\cdot (n-k)!}$
\end{itemize}

\section{Bedingte Wahrscheinlichkeit, Unabhängigkeit}
\subsection{Bedingte Wahrscheinlichkeit}
Bedingte Wahrscheinlichkeit $P(A | B)$ bedeutet, $P(A)$ unter der Bedingung, dass $B$ eintritt.
\[P(A|B) = \frac{P(A \cap B)}{P(B)}\]
Satz von Bayes:
\[P(A_i|B)=\frac{P(A_i) \cdot P(B|A_i)}{\sum^n_{k=1}{P(A_k) \cdot P(B|A_k)}}\]

\subsection{Unabhängigkeit}
Zwei Ereignisse $A$ und $B$ heißen unabhängig, wenn
\[P(A \cap B) = P(A) \cdot P(B)\]

\section{Zufallsvariablen}

\subsection{Binomialverteilung}
\[P(X=i) = \left(\begin{array}{c} n \\ i \end{array}\right) p^i(1-p)^{n-i}, i = 0,1,...,n.\]

\subsection{Poissonverteilung}
\[P(X = i) = \frac{\lambda^i}{i!} e^{-\lambda}, i=0,1,2,...\]

\section{Random}
\subsection{Tschebyscheff}
\[ P(\left| X - E(X)\right| \geq c) \leq \frac{Var(X)}{c^2}, \ c > 0\]


\end{document}