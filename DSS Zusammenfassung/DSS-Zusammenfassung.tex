\documentclass[12pt,a4paper]{article}
\usepackage[utf8]{inputenc}
\usepackage[german]{babel}
\usepackage{amsmath}
\usepackage{amsfonts}
\usepackage{amssymb}
%\usepackage{biblatex}
\usepackage{a4wide}
%\usepackage[babel,german=guillemets]{csquotes}
%\bibliography{literatur} 
\author{Maximilian Reuter}
\title{Zusammenfassung Mathe 2 für ET SS12}
\begin{document}
\maketitle
\tableofcontents
\newpage

\part{Fourier-Analyse}
\section{Basics}
Arithmetischer Mittelwert eines Singals:
\[m_x(t_1,t_2) = \frac{1}{t_2-t_1}\int\limits_{t_1}^{t_2}{x(t)dt}\]
\[m_x = \lim\limits_{T\to\infty}{\frac{1}{T}\int\limits_{T}{x(t)dt}}\]
Endliche Energie von Energiesignalen:
\[E_x(t_1,t_2) = \int\limits_{t_1}^{t_2}{x^2(t)dt}\]
\[E_x = \int\limits_{-\infty}^{\infty}{x^2(t)dt}<\infty\]
entspricht:
\[E_u = \frac{1}{R} \int\limits_{-\infty}^{\infty}{u^2(t)dt} < \infty\]
Periodizitätsbedingung:
\[u_p(t) = u_p(t+T)\]
Mittelwert:
\[m_u = \frac{1}{T}\int\limits_{t_0}^{t_0 + T}{u(t)dt}\]

\section{Fourier-Reihe mit reellen Koeffizienten}
Ansatz:
\[f(t) = \frac{a_0}{2} + \sum\limits_{n=1}^{\infty}{[a_n \cos(n\omega_1 t) + b_n \sin(n \omega_1 t)]}\]
(näheres zur Formel in der Formelsammlung...)


\end{document}