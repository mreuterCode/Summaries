\documentclass[12pt,a4paper]{article}
\usepackage[utf8]{inputenc}
\usepackage[german]{babel}
\usepackage{amsmath}
\usepackage{amsfonts}
\usepackage{amssymb}
\usepackage{a4wide}
\title{Zusammenfassung ETiT WS2011/12}
\author{Maximilian Reuter}
\begin{document}
\maketitle
\tableofcontents
\newpage

\part{Generelle wichtige Formeln}
Stromstärke:
\[I = \frac{\Delta Q}{\Delta t}\]
Widerstand in Ohm ($\Omega$):
\[R=\frac{U}{I}\]
\[R = \rho \frac{l}{A}\] 
Leitwert in Siemens ($S$):
\[G = \frac{1}{R}\]
\[G = \gamma \frac{A}{l}\]
Arbeit $W$ in Joule ($J$):
\[W = Q \cdot U = U\cdot I \cdot t\]
Leistung $P$ in Watt ($W$):
\[P = U*I\]
Potential einer durch die Arbeit $W$ verschobenen Ladung $Q$ ($\varphi_{Ursprungsort} = 0$):
\[\varphi = \frac{W}{Q}\]
Potentialdifferenz $U$ in Volt ($V$):
\[U_{ab} = \varphi_A - \varphi_B\]
Wirkungsgrad:
\[\eta = \frac{R}{R+R_i} = \frac{P_{nutz}}{P_{zu}}\]
\[0\geq \eta \geq 1\]

\part{Gleichstrom}
\section{Temperaturabhängigkeit von Widerständen}
Formel:\\
\[R = R_{20}[1+\alpha_{20}(\vartheta-\vartheta_0)]\]

\section{Die Kirchhoffschen Gesetze}
\subsection{Gesetz 1}
Alles was in einen Knoten reinfließt, muss auch wieder raus!
\[\sum_{v=1}^n I_v = 0\]

\subsection{Gesetz 2}
Die Spannungen in einem Umlauf addieren sich zu 0:
\[W_{Umlauf} = QU_{Umlauf} = 0\]
durch 
\[U_{Umlauf}= \sum_{v=1}^n U_v = 0\]

\section{Ohmscher Widerstand (Zweipole)}
\subsection{Reihenschaltung}
In Reihe geschaltete Widerstände addieren sich.
\[R_{ges} = R_1 + R_2 + R_3 + ...\]

\subsection{Parallelschaltung}
Parallelgeschaltete Widerstände lassen sich ermitteln indem man die Leitwerte addiert.
\[G_{ges} = G_1 + G_2 + G_3+...\]
mit Widerständen:
\[\frac{1}{R_{ges}} = \sum\frac{1}{R_{Teil}}\]

\subsection{Spannungsteiler}
\[U_2 = \frac{R_2}{R_1+R_2}U\]

\subsection{Stromteiler}
\[I_2 = \frac{R_1}{R_1+R_2}I\]

\subsection{Netzreduktion: Brücken-Abgleich}
$U_5 = U_3 - U_1$ soll 0 sein, daher muss $U_3 = U_1$ sein.\\
Für die Widerstände $R_1$ bis $R_4$ bedeutet das:
\[\frac{R_1}{R_1+R_2} = \frac{R_3}{R_3+R_4}\]
In der Klausur unbedingt zuerst nach Brücken suchen!!

\section{Zählpfeile}
Zählrichtung muss am Anfang willkürlich festgelegt werden und entscheidet dann am Ende über das Vorzeichen
von Strom und Spannung.\\
\begin{itemize}
\item
Bei gleichgerichtetem Strom- und Spannungspfeil: Verbraucherzählpfeilsystem\\
\item
Bei entgegengesetztem Strom- und Spannungspfeil: Erzeugerpfeilsystem\\
\end{itemize}

\section{Messung}
\subsection{Strommessung (Amperemeter)}
$R_i << R_{Last}$, da sonst eine Spannung am Amperemeter abfällt, die das Ergebnis verfälscht!\\
\\
Messbereichserweiterung: Widerstand parallel schalten: \\
\[I_{max} = \frac{G_M+G_P}{G_M}I_{M,max}\]

\subsection{Spannungsmessung (Voltmeter)}
$R_i >> R_{Last}$, da sonst Strom durch das Voltmeter fließt, der das Ergebnis verfälscht!\\
\\
Messbereichserweiterung: Widerstand in Reihe schalten: \\
\[U_{max} = \frac{R_M+R_1}{R_M}U_{M,max}\]

\section{Quellen}
\subsection{Strom- und Spannungsquelle}
Stromquelle gibt konstanten Strom bei beliebiger Spannung ab, Spannungsquelle gibt bei konstanter Spannung einen beliebigen Strom ab andersherum. Beliebig bedeutet abhängig von $R_{ges}$.

\subsection{Lineare Quelle}
Lineare Abhängigkeit zwischen $U$ und $I$
\begin{itemize}
\item Leerlaufspannung: $I = 0, U = U_0$
\item Kurzschlussstrom: $U = 0, I = I_k$
\end{itemize}
$\rightarrow$ U-I-Kennlinie!
\[U(I) = U_0 - \frac{U_0}{I_k}I\]
Innenwiderstand: 
\[R_i = \frac{U_0}{I_k}\]
Innenwiderstand durch Topologie ermitteln:\\
Wir schließen 
\begin{itemize}
\item Spannungsquellen kurz (durch leitende Verbindung ersetzen)
\item Stromquellen auf (auftrennen)
\end{itemize}
$\rightarrow$ danach einfach $R$ zwischen den beiden ausgehenden Klemmen a und b berechnen!

\subsection{Last an Quelle}
$P_{Last}$ ist maximal, wenn $R_L = R_i$\\

\section{Dioden}
\subsection{Arbeitspunkt finden}
Arbeitspunkt ist der Schnittpunkt der $U-I$-Kennlinie mit der Gerade von $I_k$ zu $U_0$.\\
Durch Veränderung der Quellenspannung erhält man eine parallele Gerade, durch Veränderung des Innenwiderstandes eine Geradenschar.\\

\subsection{Anwendungsgebiet}
Um Gleichspannungen zu stabilisieren, da eine Diode bei steigender Spannung einen steigenden Widerstand bekommt (und andersrum).

\section{Sternschaltung Dreieckschaltung}
\subsection{Dreieck-Stern-Umwandlung}
Seien $R_{2}$, $R_{13}$ und $R_{23}$ die drei Widerstände einer Dreieckschaltung, so berechnet man $R_1$, $R_2$ und $R_3$ in der Sternschaltung durch: \\
\[R_{x} = \frac{\text{Produkt der am Knoten liegenden Dreieckwiderstände}}{\text{Summe aller Dreieckwiderstände}}\]
$\rightarrow$ unabgeglichene Brücke immer durch Stern ersetzen!!

\subsection{Stern-Dreieck-Umwandlung}
\[\text{Dreieckleitwert} = \frac{\text{Produkt der zwischen den Knoten liegenden Sternleitwerte}}{\text{Summe aller Sternleitwerte}}\]

\section{Superposition}
"`Die Wirkung einer Ursache von linearen Vorgängen kann unabhängig von allen anderen Ursachen und Wirkungen betrachtet werden"'(Hermann von Helmholtz)\\
$\rightarrow$ Spannungsquellen gedanklich kurzschließen \\
$\rightarrow$ Stromquellen gedanklich unterbrechen.\\
Jeweils eine Quelle aktiv lassen und damit wie gewohnt das Netz analysieren. Die Ergebnisse werden addiert. Achtung bei Strömen/Spannungen ungleichen Vorzeichens!

\section{Umlaufanalyse}
Rezept:
\begin{enumerate}
\item
Vollständigen Baum auswählen und Stromzählpfeile für Verbindungszweige eintragen.
\item
Unabhängige Ströme über den Spalten der Matrix auftragen.
\item
Summe der Umlaufwiderstände als Hauptdiagonalelemente eintragen.
\item
Koppelwiderstände in den Nebendiagonalen eintragen, wenn gekoppelte Umläufe gegensinnig sind, negatives Vorzeichen setzen!
\item
Summe der Quellenspannungen pro Umlauf den Zeilen zuordnen.
\item
Matrix mit dem Gauss Algorithmus lösen.
\end{enumerate}

\part{Wechselstrom}
\section{Begriffe}
Periodizitätsbedingung: 
\[s(t) = s(t+T) = s(t+nT),\ n \in Z\]
Arithmetischer Mittelwert: 
\[\overline{s} = \frac{1}{T} \int\limits_{t_0}^{t_0+T}s(t)\]

\section{Zeitfunktionen}
Spannung:
\[u(t) = \hat{U} \cdot \sin\omega t\]
Stromstärke:
\[i(t) = \frac{u(t)}{R} = \frac{\hat{U}}{R}sin(\omega t)\]
Kreisfrequenz:
\[\omega = 2\pi f\]
Magnetischer Fluss:
\[\Phi = BAcos(\varphi) = BAcos(\omega t)\]
Induktionsgesetz:
\[u_i = -\frac{d\Phi}{dt} = \omega ABsin(\omega t) = \hat{U}\cdot sin(\omega t)\]

\section{Scheitel- und Formfaktoren}
\subsection{Scheitelfaktoren}
\begin{itemize}
\item Sinusschwingung: 
\[\frac{\hat{U}}{U} = \sqrt{2}\]
\item Dreieckschwingung: 
\[\frac{\hat{U}}{U} = \sqrt{3}\]
\end{itemize}

\subsection{Formfaktoren}
\begin{itemize}
\item Sinusschwingung:
\[\frac{U}{\overline{u}} = \frac{\pi}{2\sqrt{2}}\]
\item Dreieckschwingung: 
\[\frac{U}{\overline{u}} = \frac{2}{\sqrt{3}}\]
\end{itemize}

\section{Komplexe Größen}
Spannung:
\[u(t) = \hat{U} cos(\omega t +\varphi_u)\]
Komplexe Spannung:
\[\underline{u}(t) = \hat{U}e^{j(\omega t +\varphi_u)} = Re\{\underline{u}\} + j Im\{\underline{u}\}\]
Scheinwiderstand:
\[\underline{Z} = R + jX\]
Scheinleitwert:
\[\underline{Y} = G + jB\]
Scheinleistung:
\[\underline{S} = \underline{Z}I^2 = P + jQ\]
Leistungsfaktor:
\[\lambda = \frac{P}{S}=cos\varphi\]
Widerstand einer Kapazität:
\[\underline{Z}_C = -j\frac{1}{\omega C}\]
Widerstand einer Induktivität:
\[\underline{Z}_L = j\omega L\]
Wichtige Regel für Argumente(Winkel) komplexer Zahlen:
\[Winkel(\frac{1}{a+jb}) = -Winkel(a+jb) = -arctan(\frac{b}{a})\]

\section{Filter}
\subsection{Formeln}
\[1dB = 10lg\frac{P_A}{P_E} = 20lg \frac{U_A}{U_E}\]

\subsection{Tiefpass 1.Grades}
\[\frac{U_A}{U_E} = \frac{1}{\sqrt{1+j\Omega ^2}}\]
Phasenverschiebung:
\[\varphi = -\arctan(\Omega)\]
Grenzfrequenz:
\[\frac{U_A}{U_E} = \frac{1}{\sqrt(2)} \rightarrow \omega_g = \frac{\Omega_g}{RC} = \frac{1}{RC}\]

\subsection{Tiefpass 2.Grades}
\[\omega_g  = \frac{0,375}{RC}\]

\section{Komplexe Leistung}
\subsection{Wirkleistung}
\[P = \frac{\hat{U}I_{Dach}}{2}cos\varphi = UIcos\varphi = \frac{1}{2}(\underline{U}\underline{I}^* + \underline{U}^* \underline{I})\]

\subsection{Blindleistung}
\[Q = \frac{\hat{U}I_{Dach}}{2}sin\varphi = UIsin \varphi = \frac{1}{2j}(\underline{U}\underline{I}^* - \underline{U}^* \underline{I})\]
Blindleistung in einer Induktivität:
\[Q_L = X_L I^2\]
Blindleistung in einer Kapazität:
\[Q_C = -X_C I^2\]

\subsection{Komplexe Scheinleistung}
\[\underline{S} = P + jQ = UI(\cos\varphi + j\sin\varphi) = \underline{Z} I^2 = RI^2+jXI^2 = UIe^{j\varphi}\]
Winkel von $\underline{S}$:
\[\varphi = \varphi_u - \varphi_i\]

\subsection{Leistungsfaktor}
$\cos \varphi = \lambda$

\subsection{Blindleistungskompensation}
Winkel von $Z_{ges}$ soll $0$ werden.
\[Q_L + Q_C = Q_L - \omega CU^2 = 0\]
Kapazität des Kompensationskondensators (zu kompensierende Blindleistung wird fast immer induktiv verursacht):
\[\rightarrow C = \frac{Q_L}{\omega U^2}\]

\subsection{Leistungsanpassung}
\section{Transformator}
Übersetzungsverhältnis:\\
\[\ddot{u} = \frac{U_{r1}}{U_{r2}} = \frac{N_1}{N_2}\]

\section{Verlust und Streuungsfreier Transformator}
Wicklungen gleichsinnig:
\[\underline{U}_1 = j\omega L_1 \underline{I}_1 + j \omega M \underline{I}_2\]
\[\underline{U}_2 = j\omega L_1 \underline{I}_1 + j \omega M \underline{I}_1\]
Wicklungen gegensinnig:
\[\underline{U}_1 = j\omega L_1 \underline{I}_1 - j \omega M \underline{I}_2\]
\[\underline{U}_2 = - j\omega L_1 \underline{I}_1 + j \omega M \underline{I}_1\]
sonstige Formeln:
\[\frac{U_1}{U_2} = \frac{L_1}{M} = \frac{L_1}{k\sqrt{L_1 L_2}} = \sqrt{\frac{L_1}{L_2}} = \ddot{u}\]

\section{Realistischer Transformator}
Widerstände:
\[R'_{Fe} = \frac{{U_1}^2}{P_0}\]
\[\omega M = \frac{U_1}{\sqrt{{I_0}^2-(\frac{P_0}{U_1})^2}}\]
\[R_1+R'_2 = \frac{P_k}{I^2_k}\]
\[R_2 = \frac{1}{\ddot{u}^2}R'_2\]
\[\rightarrow \underline{Z}_k = R_1+R'_2+j\omega (L_1+L'_2)= \frac{\underline{U}}{\underline{I_k}}\] ( $j\omega M$ und $R_{Fe}$ sind zu vernachlässigen)
\[X_k = \sqrt{Z^2_k-R^2_k}\]
\[\omega_1 L_1 \approx \ddot{u}^2 L_2\]
Streuung:
\[\sigma = 1- \frac{M^2}{L_1 L_2}\]

\end{document}